The references for this lecture are sections 1.2-1.6 of \cite{Evans_2023}.
In this lecture we went over integrable systems and Hamiltonian torus actions.
We will denote by $M=(M^{2n},\omega)$ a symplectic manifold.

\subsubsection*{Review}

Recall from the previous lecture the Poisson bracket
\[\{-,-\}:C^\infty(M)\times C^\infty(M)\to ,\quad (F,G)\mapsto\{F,G\}:=\omega(X_F,X_G).\]
In fact $(C^\infty(M),\{-,-\})$ forms a Lie algebra.

\todo{write/order better, also only first part cartan's formula, maybe redo somehow?}

\begin{thm}[Cartan's magic formula]
    Let $N$ be any smooth manifold and $\eta\in\Omega^k(N)$ a $k$-form.
    Then for vector fields $X$ and $Y$ on $N$ the following hold.
    \begin{enumerate}
        \item $\mc L_X\eta=\dee i_X\eta+i_X\dee\eta$
        \item $i_{[X,Y]}\eta=\mc L_Xi_Y\eta-i_Y\mc L_X\eta$
    \end{enumerate}
\end{thm}

\begin{cor}
    The natural map $C^\infty(M)\to\mf X(M)$ given by $F\mapsto X_F$ is a morphism of Lie algebras.
\end{cor}

\begin{proof}
    Let $F,G\in C^\infty(M)$, we check that $X_{\{F,G\}}=[X_F,X_G]$.
    By Cartan
    \begin{align*}
        i_{[X_F,X_G]}\omega&=\mc L_{X_F}i_{X_G}\omega-i_{X_F}\mc L_{X_G}\omega\\
        &=\mc L_{X_F}i_{X_G}\omega-i_{X_F}(i_{X_g}\dee\omega+\dee i_{X_G}\omega)\\
        &=\mc L_{X_F}i_{X_G}\omega\\
        &=\dee i_{X_F}i_{X_G}\omega+i_{X_F}\dee i_{X_G}\omega
        &=-\dee\omega(X_F,X_G)
    \end{align*}
    Thus $[X_F,X_G]=X_{\{X_F,X_G\}}$.
\end{proof}

\subsubsection*{Integrable Hamiltonian system}
\todo{Check Yordan's notes here}

\begin{dfn}[Poisson commutativity]
    We say that smooth functions $F,G\in C^\infty(M)$ \emph{Poisson commute} if $\{F,G\}=0$, i.e. if $\omega(X_F,X_G)$ vanishes everywhere.
    We will refer to this simply as commuting whenever the meaning is clear from the context.
\end{dfn}

\begin{lem}[Invariance of commuting functions]
    If $F,G\in C^\infty(M)$ commute then $F$ is invariant under the flow of $X_G$.
\end{lem}

\begin{proof}
    \todo{more detail}
    We consider the behavior of $f$ on integral curves of $X_G$.
    \[\frac{\dee}{\dee t}F(\phi_{X_G}^t(x))=\dee F(X_{G,\phi_G^t(x)})=-\omega(X_{F,x},X_{G,x})=-\{F,G\}(x)=0.\]
\end{proof}

\begin{lem}[Commutativity of flows of Hamiltonian vector fields]
    The flows $\phi_F^t$ and $\phi_G^t$ commute if and only if $\{F,G\}$ is locally constant.
\end{lem}

\begin{proof}
    Recall that the flows of two vector fields commute if and only if their Lie bracket vanishes identically.
    We further recall that $[X_F,X_G]=X_{\{F,G\}}$, and by the non-degeneracy of $\omega$ the vector field $X_{\{F,G\}}$ vanishes if and only if $\omega(X_{\{F,G\}},-)=\dee\{F,G\}$ vanishes.
    But $\dee\{F,G\}$ vanishes identically exactly when $\{F,G\}$ is locally constant.
\end{proof}

\begin{dfn}[Hamiltonian $\bR^k$-action]
    Given a map 
    \[\mbf H=(H_1,\ldots,H_k):M\to\bR^k\]
    such that $\{H_i,H_j\}=0$ for all $1\le i,j\le k$ we can define an induced Hamiltonian group action $\Psi:\bR^k\acts M$ by:
    \[\Psi:\bR^k\times M\to M,\quad \Psi^t(x)=\phi_{H_1}^{t_1}\circ\cdots\circ\phi_{H_k}^{t_k}(x).\]
\end{dfn}

This definition can be extended to Lie groups.

\begin{dfn}[Hamiltonian $G$-action]
    Let $G$ be a Lie group with Lie algebra $\mf g$.
    Then a Hamiltonian $G$-action is a $G$-action such that each one-parameter subgroup $\exp(t\xi)$ acts as a Hamiltonian flow 
\end{dfn}

\begin{dfn}[Integrable system]
    We say that $\mbf H=(H_1,\ldots,H_n)$ is a complete commuting Hamiltonian system if $\{H_i,H_j\}=0$.
    If $\mc H$ is proper and has a dense set of regular values then we say that is an \emph{integrable system}.
    (A map is proper if the preimages of compact sets are compact.)
\end{dfn}

\begin{lem}[Orbits of Hamiltonian actions]
    If $\mbf H:M\to\bR^k$ generates a Hamiltonian $\bR^k$-action, then the orbits of this action are isotropic submanifolds.
    Furthermore if $M$ contains a regular point of $\mbf H$ then $k\le n$.
\end{lem}

\begin{proof}
    \todo{prove this}
\end{proof}

\begin{cor}
    If $k=n$ then the orbits of regular points are Lagrangian.
\end{cor}

\begin{proof}
    \todo{prove this}
\end{proof}