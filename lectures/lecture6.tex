No references were given for this lecture.
In this lecture we went over Delzant's theorem, displaceability, and motivation for Lagrangian submanifolds.

\todo{recollections}

\subsubsection*{Delzant's theorem}

We begin by defining Delzant polytopes.

\begin{dfn}[Rational convex polytope]
	A rational convex polytope $\mc P\subset\bR^n$ is a subset of Euclidean space which can be defined as the intersection of finitely many half-spaces:
	\[s_{\alpha,b}:=\{x\in\bR^n|\alpha\cdot x\le b\},\]
	with $\alpha\in\bZ^n$ and $b\in\bR$.
	So it is a convex polytope with rational slopes, but not necessarily rational vertices.
\end{dfn}

\begin{dfn}[Delzant polytope]
	A rational convex polytope $\mc P\subset\bR^n$ is \emph{Delzant}, if:
	\begin{enumerate}
		\item each vertex meets exactly $n$ edges,
		\item and at every vertex $v$ there exist $n$ integral vectors parallel to the edges meeting $v$ which together form an integral basis for $\bZ^n$.
	\end{enumerate}
\end{dfn}

\begin{example}
	\todo{Tikz and write out examples}
\end{example}

The following deep result gives a correspondence between symplectic toric manifolds and Delzant polytopes.
\begin{thm}[Atiyah-Guillemin-Delzant]
	\todo{there was a fourth person associated to this result, who? also what exactly is meant by same?}
	\begin{enumerate}
	\item The moment polytope of a symplectic toric manifold is Delzant.
	\item Every Delzant polytope arises as the moment polytope of some symplectic toric manifold.
	\item Any two symplectic toric manifolds with the same moment polytope are equivariantly symplectomorphic.
	\end{enumerate}
\end{thm}

\begin{remark}
	We will not prove this result, nor provide a reference at this time as it the collection of multiple disparate results.
	By sameness, we identify Delzant polytopes under the action of $GL_n(\bZ)$.
	However the construction from the previous lecture proves the second point for Delzant polytopes with integral coefficients(vertices?).
	The families of symplectic manifolds constructed in the previous lecture are in fact Kähler manifolds with integral symplectic forms, and thus embed as complex varieties.
\end{remark}
\todo{probably also translation invariance? give details on the construction}

\subsubsection*{Interlude on Lagrangian submanifolds}
Let $M=(M^{2n},\omega)$ be a symplectic manifold.
Consider the groups
\[\Ham(M)\subset\Symp_0(M)\subset\Symp(M)\subset\Diff(M),\]
where $\Symp_0(M)$ denotes the connected component of $\Symp(M)$ containing the identity, and $\Ham(M)$ denotes the symplectomorphisms given by the time $1$ maps of Hamiltonians.
So
\[\Ham(M)=\{\phi_{\mc H}^1|\mc H\text{ Hamiltonian}\}.\]
It is clear that this is a subset of $\Symp_0(M)$, as we get a path from the identity by varying the time from $0$ to $1$.
Showing that this is a group is non-trivial, but it is a group though we will not prove this.

\begin{remark}
	If $H(M,\bR)=0$, then $\Ham(M)=\Symp_0(M)$.
	This is proven in the homework.
\end{remark}

\todo{physical interpretation of Lagrangian submanifolds}

\subsubsection*{Displaceability of Lagrangians}

\begin{dfn}[Hamiltonian displaceability]
	We say that a Lagrangian submanifold $L\subset(M^{2n},\omega)$ is (Hamiltonianly) displaceable if there exists $\phi\in\Ham(M)$ such that $\phi(L)\cap L=\varnothing$.
\end{dfn}

\begin{example}
    Consider
	\todo{write and draw this.}
\end{example}

\begin{thm}[Big fiber theorem]
	Given $M^{2n}\xrightarrow\mu\mc P\subset\bR^n$ symplectic toric, there exists a point $p\in\mc P$ such that $\mu^{-1}\mc P$ is non-displaceable.
	This is sometimes referred to as symplectic rigidity.
\end{thm}

\begin{remark}
    On Floer theory,
	\todo{write a bit about Floer theory}
\end{remark}

\subsubsection*{Arnold's conjecture}

We now cover Arnold's conjecture.