No reference have been given for this lecture.
In this lecture we covered toric symplectic manifolds.

\begin{dfn}[Symplectic toric manifold]
    We say $M=(M^{2n},\omega)$ is \emph{toric} if it admits an integrable system such that the $\bR^n$ action factors through $\bR^n/\bZ^n$.
\end{dfn}

So there exists a map $\mbb H(=(H_1,\ldots,H_n)):M\to\bR^n$ which reduces to $\bR^n/\bZ^n$.

\begin{example}
    Let $M=\bC^n\cong\bR^{2n}$ and $\mbb T^n$ act on $\bC^n$ by rotation.
    So
    \[(e^{i\theta_1},\ldots,e^{i\theta_n})(z_1,\ldots,z_n)\mapsto(e^{i\theta_1}z_1,\ldots,e^{i\theta_n}z_n).\]
    Then this action is induced by the moment map
    \[\mu=\mbb H=(H_1,\ldots,H_n),\quad H_i(x_1,\ldots,x_n,y_1,\ldots,y_n)=\frac12(x_i^2+y_i^2).\]
    The image of $\mu$ is just $(\bR_{\ge0})^n$ and the preimage of a point in the open locus is just $\mbb T^n$.
\end{example}

We now move our definitions to abstract torii.

\begin{dfn}[Updated symplectic toric manifold]
    A ($k$-)torus is a $k$-dimensional connected compact Lie group.
    Such a group is non-canonically isomorphic to $(S^1)^k$.

    Let $(M^{2n},\omega)$ be a symplectic manifold, and $T$ be a $k$-torus for some $1\le k\le n$.
    Then an action $T\acts M$ is \emph{Hamiltonian} if there exists a moment map $\mu:M\to\mf t^\vee$ inducing $T\acts M$.
    Here $\mf t^\vee$ denotes the dual of the Lie algebra $T$.
    Then $M$ is toric.
    
    Given a Hamiltonian action $\Psi:T\times M\to M$ and $v\in\mf t$, we define the vector field $v^\#\in\mc X(M)$ to be
    \[v_p^\#=\left.\frac{\dee}{\dee t}\right|_{t=0}\Psi(e^{tv}p).\]
\end{dfn}

\todo{maybe something on why these are the same?}

\begin{lem}
    Given $\in\mf t$ the vector field $v^\#$ is symplectically dual to $-\dee\mu(v)$.
\end{lem}
\begin{proof}
    We work on  a basis and identify $T=\bR^k/\bZ^k$ with $\mu=(H_1,\ldots,H_k)$.
    We want to show that
    \[w(v^\#,-)=-\dee\mu(v).\]
    It suffices to check this on the basis of $\partial_{x_i}$, and we get
    \[w(\partial_{x_i}^\#,-)=-\dee H_i\]
    which is how $\partial_{x_i}^\#$ is defined.
\end{proof}
\todo{something's not right here}

\begin{lem}
    Let $M=(M^{2n},\omega)$ by symplectic and equipped with a Hamiltonian torus action $T\acts M$, with moment map $\mu$.
    Given a morphism of $\phi:T'\to T$ torii the moment map of the induced action $\mu':M\to(\mf t')^\vee$ satisfies
    \[\phi\mu'=\phi^*\circ\mu.\]
\end{lem}
\begin{proof}
    For $v\in\mf t'$ we take the pushforward $(\dee\phi(v))^\#$.
    Now $(\dee\phi(v))^\#$ is dual to $-\dee\mu(\dee\phi(v))=-\dee(\phi^*\mu)(v)$.
\end{proof}
\todo{bit more detail}

\subsubsection*{Symplectic reduction}

\begin{lem}[Linear symplectic reduction]
    Let $(V,\omega)$ be a symplectic vector space and $F\hookrightarrow V$ be a linear subspace.
    We denote by $\pi:F\to F/(F\cap F^\omega)=\bar F$ the projection map.
    Then there exists a linear symplectic form $\bar\omega$ on the quotient $\bar F$ satisfying the formula $\pi^*\bar\omega=\omega|_F$.
\end{lem}

\begin{proof}
    First we check that this defines a form.
    See that for $v\in\ker\pi=F\cap F^\omega$ and $u\in F$ we have $\omega(v,u)=0$, so taking
    \[\bar\omega(-,-)=\omega(\pi^{-1}(-),(-))\]
    does not depend on the choice of preimage.
    A form defined this way is clearly also anti-symmetric.

    To show that $\bar\omega$ is non-degenerate consider a non-zero vector $v\in\bar F$ and consider a vector $v'\in\pi^{-1}(v)$.
    As $v$ is non-zero, $v'\in F\setminus F^\omega$ so there exists some $u\in F$ such that $\bar\omega(v,\pi(u))=\omega(v',u)\neq0$.
\end{proof}

\begin{lem}[Symplectic reduction for the torus]
    Let $T\acts(M^{2n},\omega)$ be a Hamiltonian action with moment map $\mu:M\to\mf t^\vee$.
    If $c\in\mf t^\vee$ is a regular value then
    \[M\sslash{c}T:=\mu^{-1}(c)/T\quad\text{and}\quad\pi^*\bar\omega=\omega|_{\mu^{-1}(c)}\]
    define a symplectic manifold and form.
\end{lem}

\begin{proof}
    We again work with a basis and set $T=\bR^k/\bZ^k$.
    We identify $\mf t^\vee=\bR^k$.
    Because $c$ is regular $\mu^{-1}(c)$ is also a manifold.
    Given $\xi\in\mu^{-1}(c)$ we denote $\mc O\xi$ by its orbit.

    \todo{why?}
    By linear symplectic reduction is suffices to show that
    \[T_x\mc O_\xi=T_x\mu^{-1}(c)\cap(T_x\mu^{-1}(c))^\omega.\]
    \todo{finish this proof}
\end{proof}

\begin{example}
    Consider $S^1\acts\bC^n$ by $(e^{i\theta},z)\mapsto(e^{i\theta}z_1,\ldots,e^{i\theta}z_n)$.
    This is generated by $\mu=\frac12\sum_i(x_i^2+y_i^2)$.
    Consider a point $z$ such that each $z_i\neq0$.
    Then $z$ is regular and
    \[\bR^{2n}\sslash zS^1:=\mu^{-1}(z)/S^1\cong\mbb{CP}^{n-1}.\]
\end{example}

\subsubsection*{Constructing new symplectic manifolds}
Consider an exact sequence of torii
\[0\to D^k\to E=\mbb T^n\to F^{n-k}\to 0,\]
and let $\mbb T^n=E\acts\bC^n$ be the standard action of rotating every coordinate individually.
This is generated by $\mu_i=\frac12(x_i^2+y_i^2)$.
We denote $\mu_D:\bC^n\to\mf d^\vee\cong\bR^k$ be the induced map.