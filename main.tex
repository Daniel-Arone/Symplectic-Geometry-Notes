\documentclass{article} % Document class
\usepackage[english]{babel}
\usepackage{csquotes}
\usepackage{graphicx} % Required for inserting images
\usepackage{amsmath,amsfonts,amssymb,amsthm} % Most useful ams packages
\usepackage{danielenv,danielmacros} % My packages
\usepackage{hyperref} % Hyperlinks
\usepackage{enumitem} % Nicer numbered lists
\usepackage{tikz-cd,quiver} % Commutative diagrams
\usepackage{todonotes} % TODO notes
\usepackage[scr=boondox]{mathalfa}
\usepackage[backend=bibtex, style=alphabetic]{biblatex}
\addbibresource{bibliography.bib}
%\usepackage{parskip}

\hypersetup{
    colorlinks=true,
    linkcolor=blue,
    urlcolor=blue,
    citecolor=cyan,
} % Link colors

\title{\textsc{Introduction to Symplectic Geometry\\ Notes from Summer Semester 2025}}
\author{Daniel Arone}
\date{\today}

\begin{document}

\maketitle

These notes are on the course V5D6 - Introduction to Symplectic Geometry\footnote{\href{https://www.math.uni-bonn.de/~lcote/V5D6_2025.html}{https://www.math.uni-bonn.de/\textasciitilde lcote/V5D6\_2025.html}}, lectured jointly by Nathaniel Bottman and  Laurent Côté.
They are entirely my own, as are any potential mistakes in what follows.
Proceed with caution.

This course covers the basic topics in symplectic geometry.
We will not discuss Floer theory or $J$-holomorphic curves.
The main reference for this course is \cite{intro} by McDuff and Salamon.

\tableofcontents

\pagebreak

\section{Introduction to symplectic topology}
\newlecture{(10.04.25).}
In this lecture we covered linear symplectic geometry and the cotangent bundle as the first example of a symplectic manifold.
References for this lecture are sections 2.1-2.3 and 3.1 of Mc-Duff and Salamon's \emph{Introduction to Symplectic Topology}.

\begin{dfn}
    A \emph{symplectic manifold} is a pair $(M,\omega)$ of an even-dimensional smooth manifold without boundary $M$, and a closed non-degenerate $2$-form $\omega$ on $M$. 
\end{dfn}

An equivalent condition to closedness and non-degeneracy would be that the highest wedge power of $\omega$ is a volume form.
So in particular it follows that $M$ is orientable.

\subsubsection*{Linear symplectic geometry}
We now turn our attention to the simpler linear setting of symplectic geometry.

\begin{dfn}
    A symplectic vector space is a pair $(V,\omega)$ of a finite-dimensional vector space $V$ over $\bR$, and a non-degenerate skew-symmetric bilinear form $\omega$ on $V$.
\end{dfn}

\begin{prop}
    $V$ is even-dimensional.
\end{prop}

\begin{proof}
    This follows immediately from the existence of a non-degenerate skew-symmetric bilinear form.
\end{proof}

\begin{example}
    Let $V=\bR^{2n}$ and $\omega_0$ be given by
    \[\omega_0(v,w)=v^\intercal\begin{bmatrix}
        0&\operatorname{Id}_{\bR^n}\\-\operatorname{Id}_{\bR^n}&0
    \end{bmatrix}w.\]
    This shall be the canonical symplectic form on Euclidean space, and we call the specified matrix $-J_0$.
    If we denote by $x_1,\ldots,x_n,y_1,\ldots,y_n$ the standard basis on $\bR^{2n}$, then we have
    \[\omega_0(x_i,x_j)=\omega(y_i,y_j)=0\]
    for all $i,j$ and
    \[\omega(x_i,y_j)=\delta_{ij}.\]
\end{example}

\begin{dfn}
    A \emph{linear symplectomorphism} is a linear isomorphism $\psi:(V,\omega)\to(V',\omega')$ such that $\psi^*\omega'=\omega$.
    Explicitly this means that
    \[\omega(v,w)=\omega'(\psi v,\psi w).\]
\end{dfn}

\begin{remark}
    We may also consider "symplectic linear maps" in general, but because the symplectic form is non-degenerate the pull-back condition requires these to be injective.
    Thus a symplectic linear map is just a symplectomorphism onto its image.
\end{remark}

A linear symplectomorphism of $(\bR^{2n},\omega_0)$ is then represented by a matrix $A\in\operatorname{GL}_{2n}(\bR)$ such that $A^\intercal(J_0)A=J_0$.

\begin{dfn}
    A matrix $A\in\operatorname{GL}_{2n}(\bR)$ satisfying this equation is called symplectic.
    Denote the group of symplectic matrices by $\operatorname{Sp}(2n)$ or $\operatorname{Sp}(2n,\bR)$.
    \todo{compute this}
    A computation shows that $\det A=\pm1$ for symplectic $A$.
    In fact, the Pfaffian may be used to show that $\det A=1$.
\end{dfn}

Let $(V,\omega)$ be a symplectic vector space and $W\subset V$ a linear subspace.

\begin{dfn}
    The \emph{symplectic complement} of $W$ is
    \[W^\omega=\{v\in V\mid\omega(v,w)=0\text{ for all }w\in W\}.\]
    We then say that $W\subset V$ is
    \begin{itemize}
        \item \emph{isotropic} if $W\subset W^\omega$,
        \item \emph{coisotropic} if $W^\omega\subset W$,
        \item \emph{symplectic} if $W\cap W^\omega=\varnothing$, and
        \item \emph{Lagrangian} if $W=W^\omega$.
    \end{itemize}
    If $W$ is symplectic, then $(W,\omega|_W)$ is a symplectic vector space.
\end{dfn}

\begin{lem}
    We have the following two results,
    \[\dim W+\dim W^\omega=\dim V,\quad\text{and}\quad (W^\omega)^\omega=W.\]
\end{lem}

\begin{proof}
    Consider the map
    \[i_\omega:V\to V^*,\quad i_\omega(v)(w)=\omega(v,w).\]
    Then this is an isomorphism, as $\omega$ is non-degenerate.
    Now see that
    \[i_\omega(W^\omega)=\{\phi\in V^*|\phi(w)=0\text{ for all }w\in W\}=W^\perp.\]
    Because $i_\omega$ is an isomorphism, this means that
    \[\dim W+\dim W^\omega=\dim W+\dim W^\perp=\dim V.\]
    See that by definition $W\subset(W^\omega)^\omega$ and as they have the same dimension they must then be equal.
\end{proof}

\begin{cor}
The following $3$ results follow immediately,
\begin{itemize}
    \item $W$ is symplectic $\iff$ $W^\omega$ is symplectic,
    \item $W$ is isotropic $\iff$ $W^\omega$ is coisotropic, and
    \item $W$ is Lagrangian $\implies\dim W=\frac12\dim V$.
\end{itemize}
\end{cor}

\begin{thm}
    Any symplectic vector space $(V,\omega)$ of dimension $2n$ is isomorphic $(\bR^{2n},\omega_0)$.
\end{thm}

\begin{proof}
    \todo{prove this}
    It suffices to construct a \emph{symplectic basis}, i.e. a basis $u_1,\ldots,u_n,v_1,\ldots,v_n\in V$ such that
    \[\omega(u_i,u_j)=\omega(v_i,v_j)=0,\]
    for all $i,j$ and
    \[\omega(u_i,v_j)=\delta_{ij}.\]
    To construct such a basis first choose any non-zero $u_1\in V$, and because $\omega$ is non-degenerate, we can find some $v_1\in V$ such that $\omega(u_1,v_1)=1$.
    See that $u_1,v_1$ span a symplectic subspace.
    Repeating this process on the symplectic complement of their span then gets us a symplectic basis.
\end{proof}

\subsubsection*{Symplectic manifolds}
\begin{dfn}
    A symplectomorphism $(M,\omega_M)\to(N,\omega_N)$ is a diffeomorphism $\phi$ such that $\phi^*\omega_N=\omega_M$.
    We denote the symplectomorphisms of $(M,\omega_M)$ by $\operatorname{Symp}(M,\omega_M)$.
    This is a lie group.
    We say that a vector field $X\in\mf X(M)$ is symplectic if $i_X\omega$ is closed.
    We denote these  by $\chi(M,\omega)$.
\end{dfn}

\begin{remark}
    By $i_X$ we denote the interior derivative or contraction, i.e. for a $k$-form $\mu$,
    \[i_X\mu(X_1,\ldots,X_{k-1})=\mu(X,X_1,\ldots,X_{k-1}).\]
    Sometimes this is also denoted by $X\lrcorner \mu$.
\end{remark}

\begin{prop}
    Integrating symplectic vector fields results in a symplectomorphism.
    Specifically if we have a smooth families $(\phi_t)$ of smooth maps and $(X_t)$ of smooth vector fields such that
    \[\phi_0=\id\quad\text{and}\quad\frac{\dee}{\dee t}\phi_t=X_t\circ\phi_t.\]
    Then $\phi_t$ is symplectic for all $t$ if and only if $X_t$ is symplectic for all $t$.
\end{prop}

\begin{proof}
    \todo{prove this}
\end{proof}

\subsubsection*{A bit on the cotangent bundle}
We consider the cotangent bundle $T^*L$ of a closed manifold $L$.
\begin{dfn}[Canonical $1$-form]
    We denote by $\pi:T^*L\to L$ the projection of the fibers to the base space.
    Then we can take the pullback along the projection as follows.
    For $p\in L,\xi\in T^*_p(L)$ and a tangent vector $v\in T_{(p,\xi)}T^*L$, denote
    \[\lambda_{\mathrm{can}}(p,\xi)(v):=(\xi\circ\dee_p\pi)(v).\]
    We call $\lambda_{\mathrm{can}}\in\Omega^1(T^*L)$ the \emph{canonical $1$-form}.
\end{dfn}

\begin{prop}
    Let $L$ be a smooth manifold, and define $\omega_{\mathrm{can}}:=-\dee\lambda_{\mathrm{can}}\in\Omega^2(T^*L)$ the canonical symplectic form.
    Then $(T^*L,\omega_{\mathrm{can}})$ is a symplectic manifold.
\end{prop}

\begin{proof}
    It is clear that $\omega_{\mathrm{can}}$ is closed because it is exact.
    Verifying that $\omega_{\mathrm{can}}$ is non-degenerate is left to the reader.
\end{proof}

\begin{remark}
    Let $x:U\to\bR^n$ be a chart of $L$ with coordinates $x_1,\ldots,x_n$.
    At a point $p\in L$ see that the differentials $(\dee x_i)_{i=1}^n$ form a basis of the cotangent space $T_q^*L$ and denoting $y_i=\dee x_i$ we get local coordinates $(x_1,\ldots,x_n,y_1,\ldots,y_n)$ for $T^*U$.
    Then one can define the canonical $1$-form in local coordinates to be
    \[\lambda_{\mathrm{can}}=y\dee x.\]
    Then $\omega_{\mathrm{can}}=\dee x\wedge\dee y$ in local coordinates.
\end{remark}

\begin{prop}
    Let $\sigma\in\Omega^1(L)$ be a $1$-form.
    Consider it as a map $\sigma:L\to T^*L$.
    Then $\lambda_{\mathrm{can}}$ is described by the universal property that the pullback satisfyies
    \[\sigma^*\lambda_{\mathrm{can}}=\sigma,\]
    for any $\sigma$.
\end{prop}

\begin{proof}
    This is done in local coordinates on some chart $x:U\to\bR^n$.
    We write
    \[\sigma=\sum_{i=1}^n\sigma_i(x_1,\ldots,x_n)\dee x_i.\]
    Then this is a map $(x_1,\ldots,x_n)\mapsto(x_1,\ldots,x_n,\sigma_1,\ldots,\sigma_n)$.
    Thus
    \[\sigma^*\lambda_{\mathrm{can}}=\sigma^*\left(\sum_{i=1}^ny_i\dee x_i\right)=\sum_{i=1}^n\sigma_i(x_1,\ldots,x_n)\dee x_i=\sigma.\]
    To verify that this uniquely determines $\lambda$ is suffices to note that if $\sigma^*\mu=0$ for all $\sigma$, then $\mu=0$.
\end{proof}

\newlecture{(17.04.25).}
The references for this lecture are sections 3.2 and 3.3 of McDuff-Salamon.
\subsubsection*{Recap}

\todo{write this}

\begin{thm*}[Reynolds transport theorem]
    Let $\phi_t:M\to M$ be a smooth family of diffeomorphism generated by vector fields $X_t$, and $\eta_t$ a family of symplectic forms.
    Then:
    \[\frac{\dee}{\dee t}\left(\phi_t^*\eta_t\right)=\phi_t^*\left(\frac{\dee\eta_t}{\dee t}+L_{X_t}\eta_t\right)\]
\end{thm*}

\subsubsection*{Moser's trick and its consequences}
\begin{thm}[Moser's trick]
    Let $M$ be a closed smooth even-dimensional manifold.
    Let $(\omega_t)_{[t\in[0,1]}$ a family of symplectic forms on $M$ and $(\sigma_t)_{t\in[0,1]}$ a family of smooth $1$-forms such that
    \[\frac\dee{\dee t}\omega_t=\dee\sigma_t.\]
    Then there exists a family $(\psi_t)_{t\in[0,1]}$ of diffeomorphisms on $M$ satisfying
    \[\psi_t^*\omega_t=\omega_0.\]
\end{thm}

To prove this we require the following lemma.
\begin{lem}
    Let $M$ be a smooth $2n$-manifold, $Q\subset M$ a closed submanifold, and $\omega_10,\omega_1\in\Omega^2(M)$ closed forms such that for all points $q\in Q$, $\omega_{0,q}$ and $\omega_{1,q}$ are non-degenerate and agree on all of $T_qM$.
\end{lem}

\begin{proof}
    \todo{prove this}
\end{proof}

We are now equipped to prove Moser's trick.

\begin{proof}
    \todo{prove this}
\end{proof}

\subsubsection*{Consequences of Moser's trick}
\todo{short bit on Darboux's theorem}

\begin{thm*}[Darboux]
    Let $(M,\omega)$ be a symplectic $2n$-manifold.
    Then for all points $p\in M$, there exists some neighborhood $U\ni p$ such that $(U,\omega|_U)$ is symplectomorphic an open $V\subset\bR^{2n}$ equipped with the canonical symplectic form.
\end{thm*}

\begin{proof}
    \todo{prove this}
\end{proof}

\begin{thm*}[Moser's stability theorem]
    Let $M$ be a closed manifold, and $(\omega_t)_{t\in[0,1]}$ a smooth family of symplectic forms such that $[\omega_t]\in H^2_{\mathrm{dR}}(M,\bR)$ is fixed independently of $t$.
    Then there exists a family of diffeomorphism $(\psi_t)_{t\in[0,1]}$ such that
    \[\psi_t^*\omega_t=\omega_0.\]
\end{thm*}

\begin{proof}
    \todo{prove this}
\end{proof}

\begin{dfn}
    We say that a smooth submanifold $Q\subset(M,\omega)$ is symplectic, (co)isotropic, or Lagrangian if for all $q\in Q$ the tangent space $T_qQ\subset T_qM$ satisfies the corresponding property with respect to $\omega$.
\end{dfn}

Recall specifically that $T_qQ\subset T_qM$ is Lagrangian if $(T_qQ)^\omega=T_qQ$ or equivalently if $\dim Q=\dim M/2$ and $\omega_q|_{T_qQ}=0$.

\begin{example}[$T^*L,\omega_{\mathrm{can}}$]
    As previously discussed, for a manifold $L$ the cotangent bundle and canonical form give a symplectic manifold $(T^*L,\omega_{\mathrm{can}}$.
    Now the zero section $L\subset T^*L$ and the cotangent space $T_q^*L\subset T^*L$ are Lagrangian submanifolds.
    \todo{tikz this}
\end{example}

In fact we will next show that any closed Lagrangian submanifold is locally like $T^*L$.

\begin{thm*}[Weinstein neighborhood theorem]
    Let $L^n\subset(M^{2n},\omega)$ be a Lagrangian submanifold.
    Then there exist open sets
    \[L\subset U\subset M\quad\text{and}\quad L\subset V\subset T^*L,\]
    and a symplectomorphism $\phi:(V,\omega_{\mathrm{can}})\to(U,\omega|_U)$ such that $\phi|_L=\id_L$.
\end{thm*}

\begin{proof}
    \todo{prove this}
\end{proof}

\section{Dynamics and integrable systems}
\newlecture{(24.04.25).}
In this lecture we covered Hamiltonian mechanics and vector fields, as well as the Poisson bracket.
The goal of this lecture was to introduce some the historical and physical motivation for symplectic geometry.
In the notes for this lecture we omit some of the physical calculations for brevity as they will not be relevant going further.
We attempt to retain the physical and historical motivation behind symplectic geometry.

References for this lecture are sections 1.1 and 3.1 of \cite{intro}.
\emph{(Note: From this point onward the references are specifically for the third edition of the book, I am unsure which edition the previous ones refer to.)}

\subsubsection*{Hamiltonian mechanics}
Consider a physical system, the configurations of which may be described by a point $x\in\bR^n$.
For example, the positions of three celestial objects may be described by a point in $\bR^9$.
We are interested in trajectories in this space, denoted by $t\mapsto x(t)$.
Now suppose that we have a function
\[L:\bR\times\bR^n\times\bR^n\to\bR,\quad L=L(t,x,v)\]
such that trajectories are the critical points of the "action functional":
\[I(x):=\int_{t_0}^{t_1}L(t,x(t),\dot x(t))\dee t.\]
In fact, such a function models the difference between the kinetic and potential energies of the system.
\emph{(Note: In the tradition of the physicists we will abuse notation and denote $L(t,x(t),v(t))$ by $L(t,x,v)$.)}

\begin{lem}[]
    A minimal path $x:[t_0,t_1]\to\bR^n$ satisfies the Euler-Lagrange equations:
    \[\frac{\dee}{\dee t}\frac{\partial L}{\partial v}=\frac{\partial L}{\partial x},\]
    where by $\partial L/\partial v$ we refer to $(\partial L/\partial v_1,\ldots,\partial L/\partial v_n)$, and similarly for $x$.
\end{lem}

\begin{proof}
    See Lemma 1.1.1 in \cite{intro}.
\end{proof}

This is the Lagrangian formulation of mechanics.
To transform this into the Hamiltonian formulation of mechanics we apply the Legendre transformation to our coordinates.
We replace $v_i$ by $y_i$ where
\[y_i=\frac{\partial L}{\partial v_i}(x,v).\]
This is a valid coordinate transformation as long as the Legendre condition,
\[\det\left(\frac{\partial^2L}{\partial v_i\partial v_j}\right)_{ij}\neq0,\]
                      holds.
For clarity of notation we denote $G_i(t,x,y)=v_i$, and avoid reference to $v$ so that we may take $y$ as a given.

\begin{dfn}[Hamiltonian]
    We define the Hamiltonian to be
    \[H(t,x,y)=\sum_{i=1}^n y_i\cdot G_i(t,x,y)-L(t,x,G(t,x,y)).\]
    This represents the total energy within the system.
\end{dfn}

We omit the computation but one may see that
\[\frac{\partial H}{\partial x_i}=-\dot y_i\quad\text{and}\quad\frac{\partial H}{\partial y_i}=\dot x_i.\]
These are called Hamilton's equations.
Note that while the Lagrangian formalism is concerned with vectors in space and the tangent bundle, in Hamilton's reformulation we focus on the cotangent bundle.

Write $z(t)=(x(t),y(t))$.
Then we may rewrite Hamilton's equations as
\[\dot z=-J_0\nabla H(z),\quad J_0=\begin{bmatrix}
    0&-\id_n\\ \id_n&0
\end{bmatrix}.\]

\subsubsection*{The Poisson bracket}
In what follows we assume that $H$ is independent of $t$, so $H(t,x(t),y(t))=H(x(t),y(t))$.
\todo{finish this lecture}

\newlecture{}
The references for this lecture are sections 1.2-1.6 of \cite{Evans_2023}.
In this lecture we went over integrable systems and Hamiltonian torus actions.
We will denote by $M=(M^{2n},\omega)$ a symplectic manifold.

\subsubsection*{Review}

Recall from the previous lecture the Poisson bracket
\[\{-,-\}:C^\infty(M)\times C^\infty(M)\to ,\quad (F,G)\mapsto\{F,G\}:=\omega(X_F,X_G).\]
In fact $(C^\infty(M),\{-,-\})$ forms a Lie algebra.

\todo{write/order better, also only first part cartan's formula, maybe redo somehow?}

\begin{thm}[Cartan's magic formula]
    Let $N$ be any smooth manifold and $\eta\in\Omega^k(N)$ a $k$-form.
    Then for vector fields $X$ and $Y$ on $N$ the following hold.
    \begin{enumerate}
        \item $\mc L_X\eta=\dee i_X\eta+i_X\dee\eta$
        \item $i_{[X,Y]}\eta=\mc L_Xi_Y\eta-i_Y\mc L_X\eta$
    \end{enumerate}
\end{thm}

\begin{cor}
    The natural map $C^\infty(M)\to\mf X(M)$ given by $F\mapsto X_F$ is a morphism of Lie algebras.
\end{cor}

\begin{proof}
    Let $F,G\in C^\infty(M)$, we check that $X_{\{F,G\}}=[X_F,X_G]$.
    By Cartan
    \begin{align*}
        i_{[X_F,X_G]}\omega&=\mc L_{X_F}i_{X_G}\omega-i_{X_F}\mc L_{X_G}\omega\\
        &=\mc L_{X_F}i_{X_G}\omega-i_{X_F}(i_{X_g}\dee\omega+\dee i_{X_G}\omega)\\
        &=\mc L_{X_F}i_{X_G}\omega\\
        &=\dee i_{X_F}i_{X_G}\omega+i_{X_F}\dee i_{X_G}\omega
        &=-\dee\omega(X_F,X_G)
    \end{align*}
    Thus $[X_F,X_G]=X_{\{X_F,X_G\}}$.
\end{proof}

\subsubsection*{Integrable Hamiltonian system}
\todo{Check Yordan's notes here}

\begin{dfn}[Poisson commutativity]
    We say that smooth functions $F,G\in C^\infty(M)$ \emph{Poisson commute} if $\{F,G\}=0$, i.e. if $\omega(X_F,X_G)$ vanishes everywhere.
    We will refer to this simply as commuting whenever the meaning is clear from the context.
\end{dfn}

\begin{lem}[Invariance of commuting functions]
    If $F,G\in C^\infty(M)$ commute then $F$ is invariant under the flow of $X_G$.
\end{lem}

\begin{proof}
    \todo{more detail}
    We consider the behavior of $f$ on integral curves of $X_G$.
    \[\frac{\dee}{\dee t}F(\phi_{X_G}^t(x))=\dee F(X_{G,\phi_G^t(x)})=-\omega(X_{F,x},X_{G,x})=-\{F,G\}(x)=0.\]
\end{proof}

\begin{lem}[Commutativity of flows of Hamiltonian vector fields]
    The flows $\phi_F^t$ and $\phi_G^t$ commute if and only if $\{F,G\}$ is locally constant.
\end{lem}

\begin{proof}
    Recall that the flows of two vector fields commute if and only if their Lie bracket vanishes identically.
    We further recall that $[X_F,X_G]=X_{\{F,G\}}$, and by the non-degeneracy of $\omega$ the vector field $X_{\{F,G\}}$ vanishes if and only if $\omega(X_{\{F,G\}},-)=\dee\{F,G\}$ vanishes.
    But $\dee\{F,G\}$ vanishes identically exactly when $\{F,G\}$ is locally constant.
\end{proof}

\begin{dfn}[Hamiltonian $\bR^k$-action]
    Given a map 
    \[\mbf H=(H_1,\ldots,H_k):M\to\bR^k\]
    such that $\{H_i,H_j\}=0$ for all $1\le i,j\le k$ we can define an induced Hamiltonian group action $\Psi:\bR^k\acts M$ by:
    \[\Psi:\bR^k\times M\to M,\quad \Psi^t(x)=\phi_{H_1}^{t_1}\circ\cdots\circ\phi_{H_k}^{t_k}(x).\]
\end{dfn}

This definition can be extended to Lie groups.

\begin{dfn}[Hamiltonian $G$-action]
    Let $G$ be a Lie group with Lie algebra $\mf g$.
    Then a Hamiltonian $G$-action is a $G$-action such that each one-parameter subgroup $\exp(t\xi)$ acts as a Hamiltonian flow 
\end{dfn}

\begin{dfn}[Integrable system]
    We say that $\mbf H=(H_1,\ldots,H_n)$ is a complete commuting Hamiltonian system if $\{H_i,H_j\}=0$.
    If $\mf H$ is proper and has a dense set of regular values then we say that is an \emph{integrable system}.
    (A map is proper if the preimages of compact sets are compact.)
\end{dfn}

\begin{lem}[Orbits of Hamiltonian actions]
    If $\mbf H:M\to\bR^k$ generates a Hamiltonian $\bR^k$-action, then the orbits of this action are isotropic submanifolds.
    Furthermore if $M$ contains a regular point of $\mbf H$ then $k\le n$.
\end{lem}

\begin{proof}
    \todo{prove this}
\end{proof}

\begin{cor}
    If $k=n$ then the orbits of regular points are Lagrangian.
\end{cor}

\begin{proof}
    \todo{prove this}
\end{proof}

\section{Lagrangian submanifolds and toric symplectic manifolds}
\newlecture{}
No reference have been given for this lecture.
In this lecture we covered toric symplectic manifolds.

\begin{dfn}[Symplectic toric manifold]
    We say $M=(M^{2n},\omega)$ is \emph{toric} if it admits an integrable system such that the $\bR^n$ action factors through $\bR^n/\bZ^n$.
\end{dfn}

So there exists a map $\mbb H(=(H_1,\ldots,H_n)):M\to\bR^n$ which reduces to $\bR^n/\bZ^n$.

\begin{example}
    Let $M=\bC^n\cong\bR^{2n}$ and $\mbb T^n$ act on $\bC^n$ by rotation.
    So
    \[(e^{i\theta_1},\ldots,e^{i\theta_n})(z_1,\ldots,z_n)\mapsto(e^{i\theta_1}z_1,\ldots,e^{i\theta_n}z_n).\]
    Then this action is induced by the moment map
    \[\mu=\mbb H=(H_1,\ldots,H_n),\quad H_i(x_1,\ldots,x_n,y_1,\ldots,y_n)=\frac12(x_i^2+y_i^2).\]
    The image of $\mu$ is just $(\bR_{\ge0})^n$ and the preimage of a point in the open locus is just $\mbb T^n$.
\end{example}

We now move our definitions to abstract torii.

\begin{dfn}[Updated symplectic toric manifold]
    A ($k$-)torus is a $k$-dimensional connected compact Lie group.
    Such a group is non-canonically isomorphic to $(S^1)^k$.

    Let $(M^{2n},\omega)$ be a symplectic manifold, and $T$ be a $k$-torus for some $1\le k\le n$.
    Then an action $T\acts M$ is \emph{Hamiltonian} if there exists a moment map $\mu:M\to\mf t^\vee$ inducing $T\acts M$.
    Here $\mf t^\vee$ denotes the dual of the Lie algebra $T$.
    Then $M$ is toric.
    
    Given a Hamiltonian action $\Psi:T\times M\to M$ and $v\in\mf t$, we define the vector field $v^\#\in\mc X(M)$ to be
    \[v_p^\#=\left.\frac{\dee}{\dee t}\right|_{t=0}\Psi(e^{tv}p).\]
\end{dfn}

\todo{maybe something on why these are the same?}

\begin{lem}
    Given $\in\mf t$ the vector field $v^\#$ is symplectically dual to $-\dee\mu(v)$.
\end{lem}
\begin{proof}
    We work on  a basis and identify $T=\bR^k/\bZ^k$ with $\mu=(H_1,\ldots,H_k)$.
    We want to show that
    \[w(v^\#,-)=-\dee\mu(v).\]
    It suffices to check this on the basis of $\partial_{x_i}$, and we get
    \[w(\partial_{x_i}^\#,-)=-\dee H_i\]
    which is how $\partial_{x_i}^\#$ is defined.
\end{proof}
\todo{something's not right here}

\begin{lem}
    Let $M=(M^{2n},\omega)$ by symplectic and equipped with a Hamiltonian torus action $T\acts M$, with moment map $\mu$.
    Given a morphism of $\phi:T'\to T$ torii the moment map of the induced action $\mu':M\to(\mf t')^\vee$ satisfies
    \[\phi\mu'=\phi^*\circ\mu.\]
\end{lem}
\begin{proof}
    For $v\in\mf t'$ we take the pushforward $(\dee\phi(v))^\#$.
    Now $(\dee\phi(v))^\#$ is dual to $-\dee\mu(\dee\phi(v))=-\dee(\phi^*\mu)(v)$.
\end{proof}
\todo{bit more detail}

We now move on to symplectic reduction.

\begin{lem}[Linear symplectic reduction]
    Let $(V,\omega)$ be a symplectic vector space and $F\hookrightarrow V$ be a linear subspace.
    We denote by $c:F\to F/(F\cap F^\omega)$ the projection map.
    The formula $\pi^*\bar\omega=c^*\omega$ defines a linear symplectic form on the quotient $\bar F$.
\end{lem}

\newlecture{}
No references were given for this lecture.
In this lecture we went over Delzant's theorem, displaceability, and motivation for Lagrangian submanifolds.

\todo{recollections}

\subsubsection*{Delzant's theorem}

We begin by defining Delzant polytopes.

\begin{dfn}[Rational convex polytope]
	A rational convex polytope $\mc P\subset\bR^n$ is a subset of Euclidean space which can be defined as the intersection of finitely many half-spaces:
	\[s_{\alpha,b}:=\{x\in\bR^n|\alpha\cdot x\le b\},\]
	with $\alpha\in\bZ^n$ and $b\in\bR$.
	So it is a convex polytope with rational slopes, but not necessarily rational vertices.
\end{dfn}

\begin{dfn}[Delzant polytope]
	A rational convex polytope $\mc P\subset\bR^n$ is \emph{Delzant}, if:
	\begin{enumerate}
		\item each vertex meets exactly $n$ edges,
		\item and at every vertex $v$ there exist $n$ integral vectors parallel to the edges meeting $v$ which together form an integral basis for $\bZ^n$.
	\end{enumerate}
\end{dfn}

\begin{example}
	\todo{Tikz and write out examples}
\end{example}

The following deep result gives a correspondence between symplectic toric manifolds and Delzant polytopes.
\begin{thm}[Atiyah-Guillemin-Delzant]
	\todo{there was a fourth person associated to this result, who? also what exactly is meant by same?}
	\begin{enumerate}
	\item The moment polytope of a symplectic toric manifold is Delzant.
	\item Every Delzant polytope arises as the moment polytope of some symplectic toric manifold.
	\item Any two symplectic toric manifolds with the same moment polytope are equivariantly symplectomorphic.
	\end{enumerate}
\end{thm}

\begin{remark}
	We will not prove this result, nor provide a reference at this time as it the collection of multiple disparate results.
	By sameness, we identify Delzant polytopes under the action of $GL_n(\bZ)$.
	However the construction from the previous lecture proves the second point for Delzant polytopes with integral coefficients(vertices?).
	The families of symplectic manifolds constructed in the previous lecture are in fact Kähler manifolds with integral symplectic forms, and thus embed as complex varieties.
\end{remark}
\todo{probably also translation invariance? give details on the construction}

\subsubsection*{Interlude on Lagrangian submanifolds}
Let $M=(M^{2n},\omega)$ be a symplectic manifold.
Consider the groups
\[\Ham(M)\subset\Symp_0(M)\subset\Symp(M)\subset\Diff(M),\]
where $\Symp_0(M)$ denotes the connected component of $\Symp(M)$ containing the identity, and $\Ham(M)$ denotes the symplectomorphisms given by the time $1$ maps of Hamiltonians.
So
\[\Ham(M)=\{\phi_{\mc H}^1|\mc H\text{ Hamiltonian}\}.\]
It is clear that this is a subset of $\Symp_0(M)$, as we get a path from the identity by varying the time from $0$ to $1$.
Showing that this is a group is non-trivial, but it is a group though we will not prove this.

\begin{remark}
	If $H(M,\bR)=0$, then $\Ham(M)=\Symp_0(M)$.
	This is proven in the homework.
\end{remark}

\todo{physical interpretation of Lagrangian submanifolds}

\subsubsection*{Displaceability of Lagrangians}

\begin{dfn}[Hamiltonian displaceability]
	We say that a Lagrangian submanifold $L\subset(M^{2n},\omega)$ is (Hamiltonianly) displaceable if there exists $\phi\in\Ham(M)$ such that $\phi(L)\cap L=\varnothing$.
\end{dfn}

\begin{example}
    Consider
	\todo{write and draw this.}
\end{example}

\begin{thm}[Big fiber theorem]
	Given $M^{2n}\xrightarrow\mu\mc P\subset\bR^n$ symplectic toric, there exists a point $p\in\mc P$ such that $\mu^{-1}\mc P$ is non-displaceable.
	This is sometimes referred to as symplectic rigidity.
\end{thm}

\todo{write a bit about Floer theory}
\begin{remark}
    On Floer theory,
\end{remark}

\subsubsection*{Arnold's conjecture}

We now cover Arnold's conjecture.

\section{Symplectic capacities and embeddings}
\newlecture{}
\input{lectures/lecture7.tex}

\newlecture{}
\input{lectures/lecture8.tex}

\newlecture{}
\input{lectures/lecture9.tex}

\newlecture{}
\input{lectures/lecture10.tex}

\section{Hamiltonian dynamics}
\newlecture{}
\input{lectures/lecture11.tex}

\printbibliography

\end{document}
